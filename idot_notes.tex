\documentclass[11pt,fleqn]{article}

\usepackage[letterpaper, portrait, margin=0.75in]{geometry}
\usepackage{amsmath, amsbsy, amsfonts, setspace}

\setcounter{MaxMatrixCols}{10}

%\input{tcilatex}

\begin{document}


\doublespacing
%\onehalfspacing

\begin{center}
\textbf{\Large Notes on Nonlinear Dynamics and Investment}
\end{center}

These notes describe a partial equilibrium model of investment with adjustment costs on the change in investment.  These adjustment costs are asymmetric for two reasons.  First, we allow for a flexible linex function.  Second, the adjustment costs are paid as a fraction of the level of investment.  For low (high) levels of investment, the adjustment costs are small (large). 

\section{Firms}

There is a representative firm that owns its capital and uses it along with labor to produce output, $Y_t$ according to a Cobb-Douglas production function:
\begin{equation}
Y_t = A_t K^{\alpha}_t H^{1-\alpha}_t,  \label{production_function}
\end{equation}
where $A_t$ denotes technological change that affects aggregate production.
The firm purchases $X_t$ units of an investment good which is transformed into productive investment, $I_t$, according to:
\begin{equation}
I_t = V_t X_t, \label{inv_production}
\end{equation}
where $V_t$ represents investment-specific technological change. $I_t$ augments a firm's capital according to:
\begin{equation} 
K_{t+1} = (1-\delta) K_t +\left[1- \mu_t S\left(\frac{I_t}{I_{t-1}}\right) \right] I_t,  \label{capital_accumulation}
\end{equation}
where $\delta$ is the depreciation rate of existing capital and $\mu_t$ is a shock to the marginal efficiency of investment.  The investment adjustment cost, $S\left(\frac{I_t}{I_{t-1}}\right)$, is specified by the linex function:
\begin{equation}
S\left(\frac{I_t}{I_{t-1}}\right) =  b \left\{\exp\left[ a \left(\frac{I_t}{I_{t-1}}-1\right)\right]-a\left(\frac{I_t}{I_{t-1}}-1\right)-1\right\}.
\end{equation}

A firm also hires its labor input, $H_t$, in a competitive market.  It maximizes the discounted stream of profits by choosing contingency plans for $K_{t+1}$, $X_t$, and $H_t$:
\begin{equation}
\max E_0 \sum^{\infty}_{t=0} m_t  \left[Y_t - w_t H_t - X_t  \right].
\end{equation}
where $m_t$ is the firm's stochastic discount factor which it takes as given along with the real wage $w_t$. It maximizes this objective subject to equations (\ref{production_function})-(\ref{capital_accumulation}).  In an equilibrium in which labor is in supplied inelastically to firms and the supply of labor is normalized to one, the first order conditions associated with this problem are:
\begin{equation}
w_t = (1-\alpha) Y_t
\end{equation}
\begin{equation}
q_t = E_t \left\{ m_{t+1} \left[ \alpha K^{\alpha-1}_{t+1} + (1-\delta) q_{t+1} \right]  \right\} \label{K_demand}
\end{equation}
\begin{equation}
q_t \mu_t \left[ 1 - \frac{\phi_I}{2}\left(\frac{I_t}{I_{t-1}}-1\right)^2 - \phi_I \left(\frac{I_t}{I_{t-1}}-1\right) \frac{I_t}{I_{t-1}} \right]
+ E_t \left[ m_{t+1} q_{t+1} \mu_{t+1} \phi_I \left(\frac{I_{t+1}}{I_{t}}-1\right) \frac{I^2_{t+1}}{I^2_{t}} \right] =1,  \label{K_supply}
\end{equation}
where $m_{t+1} = \beta \frac{\Lambda_{t+1}}{\Lambda_t}$.  Equation (\ref{K_demand}) can be thought of as tracing out a firm's demand for capital as it implies that a firm's demand for capital, $K_{t+1}$, is inversely related to its price, $q_t$, holding all else equal.  Equation (\ref{K_supply}) can be viewed as determining the supply of capital, as this equation and the capital accumulation equation implicitly define a positive relationship between $q_t$ and $K_{t+1}$.  These relationships are easiest to see after taking a first-order approximation to the model, a task we turn to next.

\section{Approximating the Model's Dynamics}

We first begin by taking a first-approximation to the model to show how equation (\ref{K_demand}) can be viewed as the capital demand equation and equation (\ref{K_demand}) as the capital supply equation.  A first-order approximation of equations (\ref{K_demand}), (\ref{capital_accumulation}), and (\ref{K_supply}) yields:
\begin{gather}
\hat{q}_t = -\hat{r}_{t} - (1-\alpha)(1-\frac{1-\delta}{r}) \hat{K}_{t+1} + \frac{1-\delta}{r} E_t \hat{q}_{t+1}  \label{Klin_demand} \\
\hat{K}_{t+1} = (1-\delta) \hat{K}_t + \delta \left( \hat{I}_t + \hat{\mu}_t \right) \label{Klin_accumulation}  \\
\hat{q}_t + \hat{\mu}_t = \phi_I \left( \hat{I}_t - \hat{I}_{t-1} \right) - \frac{\phi_I}{r}  E_t \left( \hat{I}_{t+1} - \hat{I}_{t} \right)  \label{Klin_supply}
\end{gather}
where $r^{-1}_t = E_t m_{t+1}$.
A second order approximation of the capital demand equation yields:
\begin{gather}
\hat{q}_t + 0.5 \hat{q}^2_t  = -\hat{r}_{t} + 0.5 \hat{r}^2_{t} - (1-\alpha)(1-\frac{1-\delta}{r}) \left[ \hat{K}_{t+1} - (1-\alpha) \left(0.5 \hat{K}^2_{t+1} + \hat{r}_t \hat{K}_{t+1} \right) \right] + \\
\frac{1-\delta}{r} E_t \left[ \hat{q}_{t+1} +0.5 \hat{q}^2_{t+1} - \hat{r}_t \hat{q}_{t+1}\right] \notag 
\end{gather}
The capital supply block is given by:
\begin{gather}
\hat{K}_{t+1} + 0.5 \hat{K}^2_{t+1} = (1-\delta) \left[\hat{K}_t + 0.5 \hat{K}^2_{t} \right] + \delta \left[ \left( \hat{I}_t + \hat{\mu}_t \right) + 0.5 \left(\hat{I}_t +\hat{\mu}_t \right)^2 - 0.5 \phi_I \left( \hat{I}_t - \hat{I}_{t-1} \right)^2 \right]  \\
\hat{q}_t + \hat{\mu}_t + 0.5 \left(\hat{q}_t + \hat{\mu}_t \right)^2 = \phi_I \left[ \hat{I}_t - \hat{I}_{t-1} + \left(\hat{q}_t + \hat{\mu}_t \right) \left(  \hat{I}_t - \hat{I}_{t-1}\right) \right] + 2 \phi_I \left(  \hat{I}_t - \hat{I}_{t-1}\right)^2 - \\  
\frac{\phi_I}{r} E_t \left(1+ \hat{q}_{t+1} + \hat{\mu}_{t+1} - \hat{r}_t \right) \left( \hat{I}_{t+1} - \hat{I}_{t} \right) - \frac{5}{2} \frac{\phi_I}{r}  E_t \left( \hat{I}_{t+1} - \hat{I}_{t} \right)^2, \notag
\end{gather}
where we use certainty equivalence to replace the expectation operator where relevant in the above equations.



\end{document}
