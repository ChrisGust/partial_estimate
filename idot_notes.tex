\documentclass[11pt,fleqn]{article}

\usepackage[letterpaper, portrait, margin=0.75in]{geometry}
\usepackage{amsmath, amsbsy, amsfonts, setspace}

\setcounter{MaxMatrixCols}{10}

%\input{tcilatex}

\begin{document}


\doublespacing
%\onehalfspacing

\begin{center}
\textbf{\Large Notes on Nonlinear Dynamics and Investment}
\end{center}

These notes describe a partial equilibrium model of investment with adjustment costs on the change in investment.  These adjustment costs are asymmetric for two reasons.  First, we allow for a flexible linex function.  Second, the adjustment costs are paid as a fraction of the level of investment.  For low (high) levels of investment, the adjustment costs are small (large). 

\section{Firms}

There is a representative firm that owns its capital and uses it along with labor to produce output, $Y_t$ according to a Cobb-Douglas production function:
\begin{equation}
Y_t = A_t K^{\alpha}_t H^{1-\alpha}_t,  \label{production_function}
\end{equation}
where $A_t$ denotes technological change that affects aggregate production.
The firm purchases $X_t$ units of an investment good which is transformed into productive investment, $I_t$, according to:
\begin{equation}
I_t = V_t X_t, \label{inv_production}
\end{equation}
where $V_t$ represents investment-specific technological change. $I_t$ augments a firm's capital according to:
\begin{equation} 
K_{t+1} = (1-\delta) K_t +\left[1-  S\left(\frac{I_t}{\gamma I_{t-1}}\right) \right] \mu_t I_t,  \label{capital_accumulation}
\end{equation}
where $\delta$ is the depreciation rate of existing capital, $\mu_t$ is a shock to the marginal efficiency of investment, and $\gamma$ is the growth rate of investment.  The investment adjustment cost, $S\left(\frac{I_t}{\gamma I_{t-1}}\right)$, is specified by the linex function:
\begin{equation}
S\left(\frac{I_t}{ \gamma I_{t-1}}\right) =  b \left\{\exp \left[ a \left(\frac{I_t}{\gamma I_{t-1}}-1\right)\right]-a\left(\frac{I_t}{ \gamma I_{t-1}}-1\right)-1\right\}.
\end{equation}

A firm also hires its labor input, $H_t$, in a competitive market.  It maximizes the discounted stream of profits by choosing contingency plans for $K_{t+1}$, $X_t$, and $H_t$:
\begin{equation}
\max E_0 \sum^{\infty}_{t=0} M_t  \left[Y_t - W_t H_t - X_t  \right].
\end{equation}
where $M_t$ is the firm's stochastic discount factor which it takes as given along with the real wage $W_t$. It maximizes this objective subject to equations (\ref{production_function})-(\ref{capital_accumulation}).  In an equilibrium in which labor is in supplied inelastically to firms and the supply of labor is normalized to one, the first order conditions associated with this problem are:
\begin{equation}
W_t = (1-\alpha) Y_t
\end{equation}
\begin{equation}
Q_t = \frac{\beta_t}{\gamma_{\tilde{A}}}E_t  \left[ \alpha A_{t+1} K^{\alpha-1}_{t+1} + (1-\delta) Q_{t+1} \right]  \label{K_demand}
\end{equation}
\begin{equation}
Q_t \mu_t \left[ 1 - S\left(\frac{I_t}{\gamma I_{t-1}}\right)-S^{\prime}\left(\frac{I_t}{\gamma I_{t-1}}\right) \frac{I_t}{\gamma I_{t-1}} \right]
+ \frac{\beta_t}{\gamma_{\tilde{A}}}E_t \left[ Q_{t+1} \mu_{t+1} S^{\prime}\left(\frac{I_{t+1}}{\gamma I_{t}}\right)  \frac{I^2_{t+1}}{\gamma I^2_{t}} \right] = \frac{1}{V_t},  \label{K_supply}
\end{equation}
where $\frac{\beta_t}{\gamma_{\tilde{A}}} \equiv \frac{M_{t+1}}{M_t}$ and $\gamma_{\tilde{A}}$ is the deterministic growth rate of output defined below.
Equation (\ref{K_demand}) can be thought of as tracing out a firm's demand for capital as it implies that a firm's demand for capital, $K_{t+1}$, is inversely related to its price, $q_t$, holding all else equal.  Equation (\ref{K_supply}) can be viewed as determining the supply of capital, as this equation and the capital accumulation equation implicitly define a positive relationship between $Q_t$ and $K_{t+1}$.  We assume that both investment specific and general purpose technology are deterministic:
\begin{equation}
\frac{A_{t}}{A_{t-1}} = \gamma_A  \text{  and  }  \frac{V_{t}}{V_{t-1}}  = \gamma_V.
\end{equation}
In order to scale the economy so that it satisfies balanced growth, we define:
\begin{equation}
\tilde{A}_t = \left[ A_t V^{\alpha}_t \right]^{\frac{1}{1-\alpha}}
\end{equation}
and $y_t = \frac{Y_t}{\tilde{A}_t}$, $w_t = \frac{W_t}{\tilde{A}_t}$, and $x_t = \frac{x_t}{\tilde{A}_t}$.  Also, $i_t = \frac{I_t}{\tilde{A}_t V_t}$, $k_{t+1} = \frac{K_{t+1}}{\tilde{A}_t V_t}$, and $q_t = V_t Q_t$.  The scaled equilibrium conditions are:
\begin{equation}
q_t =  \beta_t E_t  \left[ \alpha \frac{y_{t+1}}{k_{t+1}} + \frac{(1-\delta)}{\gamma} q_{t+1} \right] \label{Kdemand_s}
\end{equation}
\begin{equation}
y_t = \left(\frac{k_t}{\gamma}\right)^{\alpha}
\end{equation}
where $\gamma = \gamma_{\tilde{A}} \gamma_V$ and $\gamma_{\tilde{A}} = \left[ \gamma_A \gamma_V^{\alpha} \right]^{\frac{1}{1-\alpha}}$.  
\begin{equation}
k_{t+1} = (1-\delta) \frac{k_t}{\gamma} + \left[ 1-  S\left(\frac{i_t}{i_{t-1}}\right) \right] \mu_t i_t
\end{equation}
\begin{equation}
q_t \mu_t \left[ 1 - S\left(\frac{i_t}{i_{t-1}}\right)-S^{\prime}\left(\frac{i_t}{i_{t-1}}\right) \frac{i_t}{i_{t-1}} \right]
+ \beta_t E_t \left[ q_{t+1} \mu_{t+1} S^{\prime}\left(\frac{i_{t+1}}{i_{t}}\right)  \frac{ i^2_{t+1}}{i^2_{t}} \right] = 1 \label{Ksupply_s}
\end{equation}
Equations (\ref{Kdemand_s})-(\ref{Ksupply_s}) are the scaled nonlinear equilibrium conditions of the model.  In the above, $S^{\prime}\left(\frac{i_t}{i_{t-1}}\right)$ is given by:
\begin{equation}
S^{\prime}\left(\frac{i_t}{i_{t-1}}\right) = ba \left\{ \exp \left[ a \left(\frac{i_t}{i_{t-1}}-1\right)\right]-1 \right\}
\end{equation}

\section{Approximating the Model's Dynamics}

Here we discuss the model's nonstochastic steady state, the linear approximation to the equilibrium conditions, and the nonlinear solution to the model.  

\subsection{Nonstochastic Steady State}

In the nonstochastic steady state, $S = S^{\prime} = 0$.  Also, $\mu_t = 1$ and $\beta_t = \beta$ for all $t$.  So, we have that $q = 1$.  From the capital accumulation equation, we have:
\begin{equation}
i = \left[ 1-\frac{1-\delta}{\gamma} \right] k
\end{equation}
From the capital demand equation, we have:
\begin{equation}
\frac{k}{y} = \frac{\alpha \beta \gamma}{\gamma-\beta (1-\delta)}
\end{equation}
The steady state level of capital is given by:
\begin{equation}
k = \gamma \left[ \frac{\alpha}{\gamma-\beta (1-\delta)} \right]^{\frac{1}{1-\alpha}}
\end{equation}

\subsection{Linear Approximation}

A first-order approximation of equations (\ref{Kdemand_s})-(\ref{Ksupply_s}) yields:
\begin{gather}
\hat{q}_t = - (1-\alpha)(1-\beta \frac{1-\delta}{\gamma}) \hat{k}_{t+1} + \frac{\beta (1-\delta)}{\gamma} E_t \hat{q}_{t+1} + \hat{\beta}_{t}\label{Klin_demand} \\
\hat{k}_{t+1} = \frac{1-\delta}{\gamma} \hat{k}_t + \left[ 1-\frac{1-\delta}{\gamma} \right]  \left( \hat{i}_t + \hat{\mu}_t \right) \label{Klin_accumulation}  \\
\hat{q}_t + \hat{\mu}_t = \phi_I \left( \hat{i}_t - \hat{i}_{t-1} \right) - \beta \phi_I  E_t \left( \hat{i}_{t+1} - \hat{i}_{t} \right)  \label{Klin_supply}
\end{gather}







\end{document}
